\documentclass{article}


\usepackage[usenames,dvips]{color}
\usepackage{colordvi}
\usepackage{html}

%% 
% Macros aimed at hog production.
% These can be tailored otherwise.
%%
\def\prompt{\#}
\def\usershell#1{\larger \prompt \framebox{\texttt{\textcolor{Blue}{#1}}} \smaller }

%% 
% Some things have to be remapped for latex2html to work or be less whiny.
% In html mode, do nothing with bShell.
% In html mode. all files must exist or we take the error.
% 	latex itself is smarter and pdfs get the message in red.
% In html, usershell has to be redone tml to get it snazzy
% Do not split the next lines in the htmlonly block here!
%% 
\begin{htmlonly}
  \usepackage{verbatim}
  \providecommand{\bShell}[2][]{}
  \providecommand{\IfFileExists}[3]{#2}
  \def\usershell#1{\begin{rawhtml}<table frame="border"><tr><td>\end{rawhtml}\prompt \begin{rawhtml}</td><td bgcolor="yellow"><b>\end{rawhtml}{\texttt{\textcolor{Blue}{#1}}}\begin{rawhtml}</b></td></tr></table>\end{rawhtml} }
\end{htmlonly}

\usepackage{graphicx}
\usepackage{amsmath}
\usepackage[DIV15]{typearea}
\usepackage{tobiShell}
\usepackage{comment}
\usepackage{relsize}



\parindent0pt
\parskip1pt
\begin{document}
\title{Docbocca}
\author{Benjamin Allan}
\maketitle
\tableofcontents

\section{Introduction}
\label{sec:intro}
We enable the execution of external commands with the
help of the \TeX{} command line option \texttt{-shell-escape} (see
section \ref{sec:usage}). Handling shell calls in tex is made easy
by use of tobiShell.sty. Basic colored region handling for both
web and pdf is handled with \textcolor{blue}{textcolor}, but coloring
embedded in verbatim blocks requires extra effort and sets off the colored areas
with latex or latex2html paragraph breaks.

Inlined shell code is written to temporary files and the external program is called with the
file name of the temporary script as the last command line argument.
Execution is immediate so that output files can be included as
part of the tex document following the directive that generates them.

The default external program is the \texttt{bash}-shell. You may
easily change the default program (e.g. to gnuplot) or specify some
other executable (env to make environment variable settings) for a given inline script.

\section{Functionality goals}
\label{sec:goals}
Our goal is to support the bocca.xml style splicing, but in a more
self-evident, extensible (and tex compatible) markup. Processing bocca.xml 
required:
\begin{enumerate}
\item unconditional shell command named block with output capture.
\item modal shell command named block with output capture.
\item block replacement with named shell command output.
\item modal block replacement with named shell command output.
\item block replacement with named sidl file.
\item sidl file-by-symbol splicing into named bocca splicer block.
\item block replacement with full method extracted from impl file of named symbol.
\item impl file-by-symbol,method splicing into named babel splicer block.
\end{enumerate}

Along with reprocessing for every babel language in use.

Equivalent tex renderings would be:
\begin{enumerate}
\item shell with redirect to capture buffer
\item (unneeded) test tex var and shell with redirect to capture buffer
\item include of named capture buffer.
\item (unneeded) include of named capture buffer.
\item include of named sidl file.
\item shell echo + shell splice
\item shell extract + include.
\item shell echo + shell splice
\end{enumerate}
But we do not need the modal versions because we can shell script
appropriately disjoint project trees from within tex.

\section{Usage}
\label{sec:usage}
\subsection{Shell with capture}
\label{sec:shell}
This is done with the bShell/eShell markup.
\paragraph{\texttt{Bash} script example}
For an example have the tutorial user do something.

% This is what we tell the user
\usershell{ls}

We can markup what the user is supposed to do any way we want;
it isn't directly executed. The bShell code is. This
lets us tell the user to do one thing but have the automation
do a slightly different invocation to get the equivalent.
This may result in a trivial amount of duplication of tex
input being required for those cases where what the user does
is identical to the shell-script-automation.
%% using the tobiShellFileName directive.
We can control the shell temporary filename for debugging purposes.
This results in:

\def\tobiShellFileName{tmp/first.sh}
\begin{latexonly}
\bShell 
ls > $TMPDIR/ls.output
db-style-sidl-input $TMPDIR/ls.output $GENDIR/ls.output.tex
\eShell
\end{latexonly}

{\input{gen/ls.output.tex}}


%% return the scratch file to the default name.
\def\tobiShellFileName{tmp/shEsc.tmp}


If something goes wrong (e.g. you didn't issue the
\texttt{-shell-escape} command line option) \LaTeX{} will complain
about the missing input file \texttt{gen/ls.output.tex} unless it
happens to be laying about from the last processing.
This is fine if you take it as an error message. 

Note the shell script executed depends on variables GENDIR and TMPDIR
assumed to be set in the environment or by the makefile. This is a
matter of taste, but is handy for keeping the tex source directory
cleaner.

\subsection{Conditionally regenerated (typically by language) text}
\label{sec:conditional}

The overall assumption is that the same latex build is invoked once
to make a complete regeneration of all included files. 
Unlike the docbook/ruby version of this process, tex+sh gives us enough
power to do it all in one pass.
Thus, to produce a final document takes 3 passes:
one to do all the input generation, one to build the table
of contents and so forth, and one to generate html. None of the
shell scripting is invoked during the html or TOC/indexing generation.
All this is done in one make invocation.

Just so this section has an example anyway, if you use some other
conditional bShell based on Victor's comment package, you may want to 
include checking for its output.

\usershell{pwd}

\begin{latexonly}
\bShell 
pwd >$GENDIR/pwd.cxx.output
\eShell 
\end{latexonly}
The response to pwd is:

%% IfFileExists works differently in latex2html than in
%% latex/pdf mode. In pdf, error red text is embedded.
%% In html mode, a missing file is an error.
\IfFileExists{gen/pwd.cxx.output}{\input{gen/pwd.cxx.output}}{
{\color{red} File \texttt{gen/pwd.cxx.output} has
  not been generated. 
}
}

If the preceding is not a directory name, it's an error message in a pdf
and the html generator will bomb.

\subsection{Inserting sidl files or output from shell commands}
\label{sec:inserting}

In some cases this is handled with plain tex include directives.
When necessary, a filter can be run over the input using bShell to
attach verbatim tags and style the user input lines (style-sidl-input.sh).
If error checking is desired, IfFileExists can be
used as demonstrated at the end of \ref{sec:conditional}
bocca is invoked to find the sidl filename.

\subsection{SIDL file splicing}
\label{sec:sidlsplice}

We want to splice into named bocca splicer block, given the full sidl symbol.
Run under the make environment, TMPDIR, GENDIR and DOCDIR
are defined.  DOCDIR is the full path of the Makefile location.
All scratch build should happen under scratch/ within DOCDIR.
If operations would conflict based on language, they should be
performed in separate subdirectories of scratch.
bocca's bin/ and DOCDIR/bin will be first in the path.

\begin{latexonly}
\def\tobiShellFileName{tmp/port.sh}
% Create project & port
% Most of this block is invariant with HOG example except the bShell line.
\bShell[env LANG=cxx PROJECT=demo SYMBOL=FunctionPort bash]
cd scratch/$LANG
bocca create project $PROJECT
cd $PROJECT
bocca create port $SYMBOL
file=`db-symbol-to-sidl $SYMBOL`
db-style-sidl-input $file $GENDIR/raw-$PROJECT.$SYMBOL.sidl.tex
\eShell
\end{latexonly}

The original sidl file looks like:
\input{gen/raw-demo.FunctionPort.sidl.tex}

\begin{latexonly}
\def\tobiShellFileName{tmp/splice-port.sh}
% Splice the port
% Most of this block is invariant with HOG example except the bShell line and user input.
\bShell[env LANG=cxx PROJECT=demo SYMBOL=FunctionPort bash]
cd scratch/$LANG/$PROJECT
file=`db-symbol-to-sidl $SYMBOL`
# put splice in file. note //tex lines disappear in the styling and destyling.
cat << EOF > $TMPDIR/$PROJECT.$SYMBOL.inp
        // DO-NOT-DELETE bocca.splicer.begin($PROJECT.$SYMBOL.methods)
//texbegin
        void   init(in array<double,1> params);
        double evaluate(in double x);
//texend
        // DO-NOT-DELETE bocca.splicer.end($PROJECT.$SYMBOL.methods)
EOF
bocca-merge \
	--from=$TMPDIR/$PROJECT.$SYMBOL.inp \
	--to=$file \
	-A "DO-NOT-DELETE bocca.splicer" -B "DO-NOT-DELETE bocca.splicer" -W
db-style-sidl-input $file $GENDIR/spliced-$PROJECT.$SYMBOL.sidl.tex
db-destyle-sidl-input $file
\eShell
\end{latexonly}

We use \texttt{bocca edit FunctionPort} to define the methods up by adding the highlighted lines:
\input{gen/spliced-demo.FunctionPort.sidl.tex}
The highlighted lines are marked by directives \texttt{//texbegin, //texend} in the splice
supplied from the tex file. These are styled to color tex/html markup when the
resulting file is prepared for tex input and are removed (destyled) from the source files
before any build or bocca operations can be applied.

\begin{latexonly}
% The bocca ansi tty color build output feature doesn't render properly in tex or html.
\bShell[env USE_COLORS=0 bash]
cd $DOCDIR/scratch/cxx/demo
bocca edit --touch FunctionPort > $TMPDIR/edit-demo.FunctionPort.log
db-style-sidl-input  $TMPDIR/edit-demo.FunctionPort.log $GENDIR/edit-demo.FunctionPort.log.tex
\eShell
\end{latexonly}

The above was be scripted with bShell and bocca-merge.
Here's the build output if user edited with bocca edit:

\input{gen/edit-demo.FunctionPort.log.tex}

\subsection{Methods extracted from Impls}
\label{sec:readmethod}
The tasks to be done here are turn a sidl name into the full path to the matching impl file,
Extract the method with bocca-extract,
Include the result.
This can be done with bShell, bocca-extract, and include, as seen below.
\begin{latexonly}
\bShell[env LANG=cxx PROJECT=demo SYMBOL=Driver bash]
cd scratch/$LANG/$PROJECT
bocca create component $SYMBOL
file=`db-symbol-to-impl Driver code`
bocca-extract -m setServices -M $file
db-style-sidl-input $PROJECT.$SYMBOL.setServices.hxx.block $GENDIR/$PROJECT.$SYMBOL.setServices.hxx.tex
\eShell
\end{latexonly}

\input{gen/demo.Driver.setServices.hxx.tex}

\subsection{Writing methods to Impls}
\label{sec:writemethod}
Next we need to replace the default exception impl of a component with some code.
First we create an implementation of function port and show the evaluate method from
it.
\begin{latexonly}
% create component and splice a method
\def\tobiShellFileName{tmp/splice-impl.sh}
%splice the port
\bShell[env LANG=cxx PROJECT=demo SYMBOL=Function FUNC=evaluate bash]
cd scratch/$LANG/$PROJECT
bocca create component -l$LANG Function --provides=FunctionPort
file=`db-symbol-to-impl Function code`
bocca-extract -m $FUNC -M $file
db-style-sidl-input $PROJECT.$SYMBOL.$FUNC.hxx.block $GENDIR/raw-$PROJECT.$SYMBOL.$FUNC.hxx.tex
cat << EOF > $TMPDIR/$PROJECT.$SYMBOL.$FUNC.inp
  // DO-NOT-DELETE splicer.begin($PROJECT.$SYMBOL.$FUNC)
//texbegin
  return 4.0/(1+x*x);
//texend
  // DO-NOT-DELETE splicer.end($PROJECT.$SYMBOL.$FUNC)
EOF
bocca-merge \
	--from=$TMPDIR/$PROJECT.$SYMBOL.$FUNC.inp \
	--to=$file \
	-A "DO-NOT-DELETE splicer" -B "DO-NOT-DELETE splicer" -W
bocca-extract -m $FUNC -M $file
db-style-sidl-input $PROJECT.$SYMBOL.$FUNC.hxx.block $GENDIR/spliced-$PROJECT.$SYMBOL.$FUNC.hxx.tex
db-destyle-sidl-input $file
\eShell
\end{latexonly}

\input{gen/raw-demo.Function.evaluate.hxx.tex}

And the the user bocca edits it to match:
\input{gen/spliced-demo.Function.evaluate.hxx.tex}

We can show the output as we did at the end of \ref{sec:sidlsplice}.

\subsection{Styling make output}
\label{sec:makeoutput}
Here's a prettified example of make output.
\begin{latexonly}
\def\tobiShellFileName{tmp/shEsc.tmp}
\bShell[env LANG=cxx PROJECT=demo bash]
cd scratch/$LANG/$PROJECT
./configure && make USE_COLORS=0 > $TMPDIR/build.log 2>&1
db-style-build-output $TMPDIR/build.log $GENDIR/build.log.tex
\eShell
\end{latexonly}

\input{gen/build.log.tex}

\subsection{Running other shells}
\label{sec:othershell}
Other commands than the \texttt{bash} shell can be used. See tobiShell.tex
for more examples with gnuplot.

\end{document}
%%% Local Variables: 
%%% mode: latex
%%% LaTeX-command-style: (("." "latex --shell-escape"))
%%% TeX-master: t
%%% End: 
